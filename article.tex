%  article.tex (Version 2.81, released 24 September 2003)
%  Article to demonstrate format for SPIE Proceedings
%  Special instructions are included in this file after the
%  symbol %>>>>
%  Numerous commands are commented out, but included to show how
%  to effect various options, e.g., to print page numbers, etc.
%  This LaTeX source file is composed for LaTeX2e,
%  not the older LaTeX version 2.09, as previous versions were.

%  The following commands have been added in the SPIE class
%  file (spie.cls) and will not be understood in other classes:
%  \supit{}, \authorinfo{}, \skiplinehalf, \keywords{}
%  The bibliography style file is called spiebib.bst,
%  which replaces the standard style unstr.bst.

%% \documentclass[]{./cls/spie}  %>>> use for US letter paper
\documentclass[a4paper]{./cls/spie}  %>>> use this instead for A4 paper
\addtolength{\voffset}{9mm}   %>>> moves text field down

%  The following command loads a graphics package to include images
%  in the document. It may be necessary to specify a DVI driver option,
%  e.g., [dvips], but that may be inappropriate for some LaTeX
%  installations.
% \usepackage{setspace}
\usepackage[dvips,
    bookmarks,
    bookmarksopen = false,
    bookmarksnumbered = true,
    breaklinks = true,
    linktocpage,
    pagebackref,
    colorlinks = true,
    linkcolor = black,
    urlcolor  = blue,
    citecolor = red,
    anchorcolor = green,
    pdftitle={Advanced HTML5 features in Tizen Web Platform},
    pdfsubject={Conference Proposal},
    pdfkeywords={HTML5, WebRTC,getUserMedia},
    pdfauthor={Soo-Hyun Choi},
    hyperindex = true]{hyperref}
% \usepackage[]{color,graphicx}
\usepackage{color,graphicx}
\usepackage{verbatim}
\usepackage[T1]{fontenc}
\usepackage{amssymb}
\usepackage{amsmath}
\usepackage{amsfonts}
\usepackage{subfigure}
\usepackage{listings}
\usepackage{url}
\usepackage{fancyhdr}
\usepackage{multirow}
\usepackage{pifont}

\usepackage{afterpage}
\usepackage{color}
\usepackage{array}
\usepackage{float}
% \usepackage[ulem=normalem]{changes}
\usepackage{changebar}
\setcounter{changebargrey}{0}


% \title{WebRTC (Real-time Communications) for Tizen Platform}
\title{Advanced HTML5 features in Tizen Web Platform}

%>>>> The author is responsible for formatting the
%  author list and their institutions.  Use  \skiplinehalf
%  to separate author list from addresses and between each address.
%  The correspondence between each author and his/her address
%  can be indicated with a superscript in italics,
%  which is easily obtained with \supit{}.

% \author{Author 1\supit{a} and Author 2\supit{b}

\author{{\normalsize{\bf Soo-Hyun Choi and Wonsuk Lee}}
\skiplinehalf
{\sf {\normalsize Next Generation S/W R\&D Group\\ Samsung Electronics, Co.,
Ltd.}}\\
{\tt {\small \{sh9.choi, wonsuk11.lee\}@samsung.com}}
% \supit{a}Affiliation1, Address, City, Country; \\
% \supit{b}Affiliation2, Address, City, Country
}

%>>>> Further information about the authors, other than their
%  institution and addresses, should be included as a footnote,
%  which is facilitated by the \authorinfo{} command.

% \authorinfo{Further author information: (Send correspondence to
% A.A.A.)\\A.A.A.: E-mail: aaa@tbk2.edu, Telephone: 1 505 123 1234\\  B.B.A.:
% E-mail: bba@cmp.com, Telephone: +33 (0)1 98 76 54 32}
%%>>>> when using amstex, you need to use @@ instead of @


%%%%%%%%%%%%%%%%%%%%%%%%%%%%%%%%%%%%%%%%%%%%%%%%%%%%%%%%%%%%%
%>>>> uncomment following for page numbers
\pagestyle{plain}
%>>>> uncomment following to start page numbering at 301
%\setcounter{page}{301}

\begin{document}
\maketitle

%%%%%%%%%%%%%%%%%%%%%%%%%%%%%%%%%%%%%%%%%%%%%%%%%%%%%%%%%%%%%
\begin{abstract}

% In the advancement and proliferation of HTML5 technologies, WebRTC (Real-time
% Communications)\cite{webrtc.w3c} has been gaining a remarkable interest among
% web developers. It basically enables web browsers to have Real-Time 
% Communications (RTC) capabilities using relatively simple JavaScript APIs. Until 
% recently, RTC applications typically have required to build complex systems 
% which involves not-so-trivial audio/video engineering works in real world 
% environment. These difficulties have been a huge barrier for individual 
% developers to create various types of RTC-capable applications, where WebRTC has 

In the proliferation of the recent HTML5 technologies, web platforms, especially 
mobile web platforms, have been achieving remarkable advancements in terms of 
gaining supported features via relatively simple JavaScript APIs. To reflect 
this tech movement, many browser vendors, including Mozilla, Chrome, Opera, and 
Safari, have already started to implement them in their browsers, in which 
uncannily introducing HTML5 to be a gist of those web platforms. At the same 
time, HTML5 features are actively discussed for standardization to better take 
into account the recent development activities at a number of W3C's working 
groups.

Tizen platform also implements many of the attractive HTML5 features such as
WebRTC, WebAudio API, File System API, File API, and various Device APIs 
(vibration API, battery status API, Network Information API, etc) of which are 
almost completed as of now. Amongst these new features, we would like to take a 
chance to demonstrate our WebRTC implementation using a sample peer-to-peer 
application in this conference; the sample app calls a series of JavaScript APIs 
associated with {\sf getUserMedia} and {\sf PeerConnection API} defined in the 
WebRTC spec\cite{webrtc.w3c}.

Lastly, we would like to talk, at the conference, about Tizen's future direction 
in extending the remaining and upcoming HTML5 features that can drive it to be a 
more popular web-based platform.




\end{abstract}

%>>>> Include a list of keywords after the abstract

\keywords{{\small\sf HTML5, WebRTC, getUserMedia, WebAudio API}}


%%%%%%%%%%%%%%%%%%%%%%%%%%%%%%%%%%%%%%%%%%%%%%%%%%%%%%%%%%%%%
% \acknowledgments     %>>>> equivalent to \section*{ACKNOWLEDGMENTS}
%
% This unnumbered section is used to identify those who have aided the authors in
% understanding or accomplishing the work presented and to acknowledge sources of
% funding.

%%%%%%%%%%%%%%%%%%%%%%%%%%%%%%%%%%%%%%%%%%%%%%%%%%%%%%%%%%%%%


%%%%% References %%%%%

\bibliographystyle{./cls/spiebib}   %>>>> makes bibtex use spiebib.bst
% \bibliographystyle{abbrv}   %>>>> makes bibtex use spiebib.bst
\bibliography{article}   %>>>> bibliography data in report.bib

\end{document}
